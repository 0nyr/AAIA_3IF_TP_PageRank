% !TeX spellcheck = fr-FR, en-US
% !TeX encoding = UTF-8
% !TEX program = pdflatex

% VSCODE word wrap: ALT + Z
% COMPILE WITH:
% `latexmk`
% latexmk -pdf main.tex
% You need pdflatex and biber (in all TeXLive distributions)

\documentclass[11pt]{article} % text width
\usepackage[utf8]{inputenc} % encode text to utf8

% Set the label for itemize globally, if desired
\usepackage{enumitem}
\setlist[itemize]{label=\textbullet}

% paragraph formatting: https://www.overleaf.com/learn/latex/Paragraph_formatting
\setlength{\parindent}{1em}
\setlength{\parskip}{1em}

% better language support
\usepackage[english, french]{babel}

% use pdflatex
\usepackage[T1]{fontenc} % font encoding
\usepackage[a4paper, margin=2cm, head=18.0pt]{geometry} % set margins to 1.5 cm
\usepackage{graphicx}% for graphics
\usepackage[onehalfspacing]{setspace}
\usepackage{tocbasic}
\usepackage{booktabs}
\usepackage{multicol}
\usepackage{multirow}
\usepackage[]{scrlayer-scrpage}
\usepackage[titletoc]{appendix}
\usepackage{comment}

\usepackage{dirtree} % directory tree
\usepackage{float} % float images
\usepackage{microtype} % better text formatting, avoid one word on a new line for example
\usepackage{mathtools} % math tools
\usepackage{amssymb} % math symbols
\usepackage{algorithm} % algorithm
% \usepackage{algorithmic}  % old, use algorithmicx from algpseudocode
\usepackage{algpseudocode} % pseudocode

% fix Font shape `T1/fve/m/sc' undefined on \Procedure
\renewcommand{\algorithmicprocedure}{\textbf{Procedure}}

\usepackage{tikz} % for drawing graphs
\usetikzlibrary{arrows.meta} % for arrow styles
%\usetikzlibrary{positioning} % for relative positioning of nodes

% quotes and bibliography: https://www.overleaf.com/learn/latex/Typesetting_quotations
\usepackage{csquotes}
\usepackage[
    % Use the french style of quotes, which are more visibly distinct
    left = \flqq{},% 
    right = \frqq{},% 
    leftsub = \flq{},% 
    rightsub = \frq{} %
]{dirtytalk}
\DeclareQuoteStyle{english}{\glqq}{\grqq}{\glq}{\grq}

% \usepackage[
%     backend=biber,
%     style=numeric,
%     sorting=none
% ]{biblatex}
%\usepackage[backend=biber, style=numeric, defernumbers=true, language=american]{biblatex}
\usepackage[backend=biber, style=numeric, sorting=none, defernumbers=true, language=american]{biblatex}
% add commands for automatic cite/uncite distinction
\DeclareBibliographyCategory{cited}
\AtEveryCitekey{\addtocategory{cited}{\thefield{entrykey}}}
\addbibresource{biblio.bib} % bibliography
\nocite{*} % all references

\newcommand{\ts}{\textsuperscript} % superscript for 2nd or XIXème

\pagenumbering{roman} % set page numbering of front matter sections

% use acronyms and glossaries
% toc: add glossary to table of contents
\usepackage{hyperref}
\usepackage{xurl}
\usepackage[acronym, toc]{glossaries} 
\makeglossaries
\newglossaryentry{memgraph}
{
  name=memory graph,
  description={A memory graph, or \textit{memgraph} is a graph representation of a memory dump. This graph can be a graph of blocks, where each node in the graph corresponds to a block of 8 bytes in the heap dump and each edge corresponds to a pointer from one block to another, or describes which blocks are part of a chunk whose root note is a Chunk Header Node. It can also be a graph of chunks (only CHNs), where each node in the graph corresponds to a chunk in heap dump and each edge corresponds to a pointer from one object to another.}
}

\newglossaryentry{nodes}
{
  name=nodes,
  description={A node is an entity in a graph, it can be a person, a place, a thing, or any other entity.}
}
\newglossaryentry{chn}{
  name={CHN},
  description={Chunk Header Node. This is a node whose bytes have been identified as a data structure header. In the graph, this node is the root node of an malloc-allocated memory chunk.}
}

\newglossaryentry{pn}{
  name={PN},
  description={Pointer Node. This is a node whose bytes have been identified as a pointer.}
}

\newglossaryentry{kn}{
  name={KN},
  description={Key Node. This is a node whose bytes have been identified as a key. This identification relies both on the annotations and some verification checks.}
}

\newglossaryentry{vn}{
  name={VN},
  description={Value Node. These are all blocks that have not been identified. It is the default node type.}
}

\newacronym{kg}{KG}{Knowledge Graph}
\newacronym{foss}{FOSS}{Free and Open Source Software}
\newacronym{rdf}{RDF}{Resource Description Framework}
\newacronym{rdfs}{RDFS}{Resource Description Framework Schema}
\newacronym{owl}{OWL}{Web Ontology Language}
\newacronym{ml}{ML}{Machine Learning}
\newacronym{dl}{DL}{Deep Learning}
\newacronym{fe}{FE}{Feature Evaluation}
\newacronym{nlp}{NLP}{Natural Language Processing}
\newacronym{ke}{KE}{Knowledge Engineering}
\newacronym{del}{DEL}{Directed Edge-labelled Graphs}
\newacronym{er}{ER}{Entity Resolution}
\newacronym{qa}{QA}{Quality Assurance}
\newacronym{sparql}{SPARQL}{SPARQL Protocol and RDF Query Language}
\newacronym{ssh}{SSH}{Secure Shell Protocol}
\newacronym{os}{OS}{Operating System}
\newacronym{vm}{VM}{Virtual Machine}
\newacronym{ddos}{DDoS}{Distributed Denial of Service}
\newacronym{ess}{ESS}{Estimated Security Strength}
\newacronym{vmi}{VMI}{Virtual Machine Introspection}
\newacronym{smote}{SMOTE}{Synthetic Minority Over-sampling Technique}
\newacronym{svm}{SVM}{Support Vector Machine}
\newacronym{knn}{KNN}{K-Nearest Neighbors}
\newacronym{rf}{RF}{Random Forest}
\newacronym{mlp}{MLP}{Multi-Layer Perceptron}
\newacronym{relu}{ReLU}{Rectified Linear Unit}
\newacronym{sgd}{SGD}{Stochastic Gradient Descent}
\newacronym{ai}{AI}{Artificial Intelligence}
\newacronym{pca}{PCA}{Principal Component Analysis}
\newacronym{lda}{LDA}{Linear Discriminant Analysis}
\newacronym{tsne}{t-SNE}{t-distributed Stochastic Neighbor Embedding}
\newacronym{msb}{MSB}{Most Significant Bit}
\newacronym{lsb}{LSB}{Least Significant Bit}

%deep learning
\newacronym{lstm}{LSTM}{Long Short-Term Memory}
\newacronym{gru}{GRU}{Gated Recurrent Units}
\newacronym{rnn}{RNN}{Recurrent Neural Networks}
\newacronym{cnn}{CNN}{Convolutional Neural Networks}
\newacronym{rcnn}{RCNN}{Recurrent Convolutional Neural Network}
\newacronym{gnn}{GNN}{Graph Neural Network}
\newacronym{gcn}{GCN}{Graph Convolutional Networks}
\newacronym{llm}{LLM}{Large Language Model}

% INSA
\newacronym{insa}{INSA}{Institut National des Sciences Appliquées}
\newacronym{aaia}{AAIA}{Algorithmique Avancée pour l'Intelligence Artificielle et les graphes}

%\input{glossaries.tex} % acronyms definitions, failed to make in work on a separate file!!!

% custom commands
% escape char in latex: https://tex.stackexchange.com/questions/34580/escape-character-in-latex
% horizontal spacing: https://tex.stackexchange.com/questions/74353/what-commands-are-there-for-horizontal-spacing/74354
\newcommand{\p}{\texttt{+}} % small unary plus
\newcommand{\doublep}{\texttt{++}} % double small unary plus
\newcommand{\m}{\texttt{-} \space} % small unary minus
\newcommand{\doublem}{\texttt{-}\texttt{-} \space} % double small unary minus

% title and section formatting
\usepackage{titlesec}

\setcounter{tocdepth}{3} % set depth of table of contents
\setcounter{secnumdepth}{3}  % Numbering depth of sections

% ------------------------------ code ------------------------------
\usepackage{listings} % code listings

% code listing style
\usepackage{bera}% optional: just to have a nice mono-spaced font
\usepackage{listings}
\usepackage{xcolor}

\definecolor{eclipseStrings}{RGB}{42,0.0,255}
\definecolor{eclipseKeywords}{RGB}{127,0,85}
\definecolor{punctuationcolor}{rgb}{0.5,0,0}
\definecolor{delimcolor}{rgb}{0,0.5,0}
\definecolor{red}{rgb}{1,0,0}
\colorlet{numb}{magenta!60!black}

\lstset{
  language=bash,
  basicstyle=\ttfamily\small,
  breaklines=true,
  frame=single,
  numbers=left,
  numberstyle=\tiny,
  showstringspaces=false,
  tabsize=4,
  captionpos=b
}

\lstdefinestyle{json}{
  basicstyle=\ttfamily\small,
  breaklines=true,
  postbreak=\mbox{\space},
  columns=fullflexible,
  showstringspaces=false,
  commentstyle=\color{gray},
  keywordstyle=\color{black},
  numberstyle=\tiny\color{gray},
  numbers=left,
  frame=single,
  captionpos=b
}

\lstdefinestyle{text}{
  basicstyle=\ttfamily\small,
  breaklines=true,
  postbreak=\mbox{\space},
  columns=fullflexible,
  showstringspaces=false,
  commentstyle=\color{gray},
  keywordstyle=\color{black},
  numberstyle=\tiny\color{gray},
  numbers=left,
  frame=single,
  captionpos=b
}

\lstdefinestyle{hexdump}{
  basicstyle=\ttfamily\small,
  breaklines=true,
  postbreak=\mbox{\space},
  columns=fullflexible,
  showstringspaces=false,
  commentstyle=\color{gray},
  keywordstyle=\color{black},
  numberstyle=\tiny\color{gray},
  numbers=left,
  frame=single,
  captionpos=b
}

\lstdefinestyle{rust}{
  basicstyle=\ttfamily\small,
  breaklines=true,
  postbreak=\mbox{\space},
  columns=fullflexible,
  showstringspaces=false,
  commentstyle=\color{gray},
  keywordstyle=\color{black},
  numberstyle=\tiny\color{gray},
  numbers=left,
  frame=single,
  captionpos=b
}

\usepackage{hyphenat} % fix "overfull hbox" with slicing words using hyphenation
\hyphenation{hy-phen-a-tion} % indicate all 3 permissible hyphenation points

% where to put all images and figures
\graphicspath{{img/}}

% customize the header and footer of the document
\usepackage{scrlayer-scrpage}
\clearpairofpagestyles
\cfoot[\pagemark]{\pagemark}

% !TeX spellcheck = fr_FR, en_US
% !TeX encoding = UTF-8

% document info
\newcommand{\thetitle}{AAIA TP 3IF - PageRank}
\newcommand{\theauthor}{Florian Rascoussier \and Christine Solnon}

\title{\thetitle}
\author{\theauthor}
\date{\today} % title information

% document content
\begin{document}

% -- Title
\maketitle
% WARN: It's not really possible to add an image before the title, without a real title page
{ % WARN: The centering command needs its own scope
  \centering
  \includegraphics[width=7cm, height=1.5cm]{Logo_INSA.png}\\
}
%%%%%%%%%%%%%%%%%%%%%%%%%%%%%%%%%%%%%%%%%%%%%%%%%%%%%%%%%%%%%%%%%%%%%%%%%%%%%%%%%%%%%%%%%

% -- Abstract, Acknowledgements, Introduction...
% !TeX spellcheck = fr
% !TeX encoding = UTF-8

% -- Introduction
\section*{Introduction}

L'algorithme PageRank, développé par Larry Page et Sergey Brin à la fin des années 1990, est au cœur du moteur de recherche original de Google. Il a révolutionné la manière dont les informations sont récupérées sur le vaste espace qu'est l'Internet. L'idée fondamentale derrière PageRank est simple et élégante : mesurer l'importance d'une page web en fonction du nombre et de la qualité des liens qui y pointent. En pratique, cela revient à considérer le web comme un graphe, où chaque page est un sommet et chaque lien est une arête dirigée. L'algorithme effectue ensuite une marche aléatoire sur ce graphe pour déterminer la probabilité qu'un "surfeur" aléatoire arrive à une page donnée, cette probabilité servant d'indicateur de l'importance ou du "rang" de la page.

Toutefois, la mise en œuvre effective de PageRank dans un contexte réel est loin d'être triviale. Elle doit tenir compte de nombreux défis techniques et théoriques, notamment comment traiter les sommets absorbants ou comment garantir la convergence de l'algorithme vers une distribution stable. Ce document vise à explorer certains de ces défis, en particulier comment transformer une matrice de transition classique en une matrice stochastique ergodique, et démontre pourquoi et comment de telles transformations préservent l'essence de l'algorithme PageRank tout en garantissant ses propriétés mathématiques fondamentales.

Nous encourageons vivement tous les étudiants et lecteurs à participer activement à l'amélioration de cette correction et du code associé. Si vous trouvez des erreurs, des ambiguïtés ou des omissions, ou si vous avez des suggestions pour améliorer le contenu, n'hésitez pas à soumettre une Pull-Request sur le GitHub de la correction à l'adresse suivante : \url{https://github.com/0nyr/AAIA_3IF_TP_PageRank}. Votre contribution est précieuse et aidera à rendre ces ressources plus précises et utiles pour tous.

%%%%%%%%%%%%%%%%%%%%%%%%%%%%%%%%%%%%%%%%%%%%%%%%%%%%%%%%%%%%%%%%%%%%%%%%%%%%%%%%%%%%%%%%%

\newpage

% francisation des noms de tableaux, figures, etc.
\renewcommand{\contentsname}{Table des Matières}
\def\lstlistingautorefname{Cd.}
\renewcommand{\lstlistingname}{Code}
\renewcommand{\lstlistlistingname}{Codes et programmes}

% -- Table of contents
\tableofcontents
\listoffigures
% \listoftables
\lstlistoflistings
% \newpage

%%%%%%%%%%%%%%%%%%%%%%%%%%%%%%%%%%%%%%%%%%%%%%%%%%%%%%%%%%%%%%%%%%%%%%%%%%%%%%%%%%%%%%%%%
\pagenumbering{arabic} % reset page numbering of main matter sections

\newpage

% -- Content
% !TeX spellcheck = fr
% !TeX encoding = UTF-8

% -- Introduction

% -- Exercice 1
\section{Exercice 1 : Marche aléatoire et Matrice de Transition dans PageRank}

\textit{Montrez que pour tout $k \geq 1$ et tout couple de sommets $(i, j) \in S \times S, M^k[i][j]$ est égal à la probabilité d'arriver sur la page j en k clics à partir de la page i.}

Pour comprendre comment la matrice de transition $M$ se rapporte à la probabilité d'arriver sur une page donnée en $k$ clics, considérons d'abord ce que signifie une marche aléatoire sur le graphe du web $G = (S, A)$.

\subsection{Rappels et explications}
Les propriétés étudiées dans le cours\footnote{Voir le cours d'\acrshort{aaia}, section 3.1 p11. \url{http://perso.citi-lab.fr/csolnon/supportAlgoGraphes.pdf}} sur la théorie des graphes aborde les notions de marche aléatoire et de matrice de transition \cite{Solnon2016coursAAIA}. Ces notions sont essentielles pour comprendre l'algorithme PageRank.

\subsubsection{Marche aléatoire sur le graphe}

Dans le contexte de l'algorithme PageRank, la marche aléatoire représente un utilisateur naviguant sur le Web en cliquant sur des liens de manière aléatoire. À chaque étape:
\begin{itemize}
    \item L'utilisateur est sur une page $i$ (un sommet dans le graphe).
    \item Il clique sur un lien sortant au hasard, le menant à une page $j$ (un autre sommet).
\end{itemize}

\subsubsection{Matrice de transition $M$}

La matrice de transition $M$ reflète la probabilité de passer d'une page à une autre en un seul clic. Ainsi, $M[i][j]$ représente la probabilité de passer de la page $i$ à la page $j$. Elle est définie comme :

\begin{itemize}
    \item $M[i][j] = \frac{1}{d^+(i)}$ si un lien (arc) existe de $i$ à $j$ (c'est-à-dire $i \rightarrow j$).
    \item $M[i][j] = 0$ sinon.
\end{itemize}

Où $d^+(i)$ est le nombre de liens sortants de la page $i$ (son demi-degré extérieur).

\subsubsection{Puissances de la matrice de transition $M^k$}

Lorsque l'on considère $M^k$, avec $k > 1$, on regarde le processus de navigation après $k$ clics. Voici comment cela fonctionne :

\begin{itemize}
    \item $M^1 = M$ représente la probabilité de passer d'une page à une autre en un seul clic.
    \item $M^2 = M \times M$ représente la probabilité de passer d'une page à une autre en deux clics.
    \item Et ainsi de suite, jusqu'à $M^k$, qui représente la probabilité de passer d'une page à une autre en $k$ clics.
\end{itemize}

Pour montrer que $M^k[i][j]$ est égal à la probabilité d'arriver sur la page $j$ en $k$ clics à partir de la page $i$, on utilise la propriété des marches aléatoires et des matrices de transition. La démonstration informelle peut être faite par induction sur $k$ :

\begin{itemize}
    \item \textbf{Initialisation (k=1)} : Par définition, $M[i][j]$ est la probabilité d'aller de $i$ à $j$ en un clic.
    \item \textbf{Récurrence} : Supposons que $M^{k-1}[i][j]$ représente la probabilité d'aller de $i$ à $j$ en $k-1$ clics. $M^k$ est alors le produit de $M$ et $M^{k-1}$, ce qui signifie que chaque élément $M^k[i][j]$ est la somme des probabilités de passer de $i$ à un certain intermédiaire $m$ en un clic (selon $M[i][m]$) et puis d'aller de $m$ à $j$ en $k-1$ clics (selon $M^{k-1}[m][j]$). Cela reflète la nature cumulative des probabilités dans les marches aléatoires.
\end{itemize}

\subsection{Démonstration formelle par récurrence}

Nous cherchons à démontrer que pour une matrice de transition $M$ d'une marche aléatoire sur le graphe du web, $M^k[i][j]$ représente la probabilité d'arriver sur la page $j$ en $k$ clics à partir de la page $i$.

\paragraph*{Rappel}
En Mathématique, démonter qu'une proposition $P(n)$ est vraie pour tout $n \in \mathbb{N}$ signifie démontrer que $P(1)$ est vrai quelle que soit la valeur de $n$. Dans ce cas, une preuve par récurrence sur $k$ est appropriée.

\subsubsection{Notations et Déclaration des Variables}

\begin{itemize}
    \item $M \in \mathbb{R}^{|S| \times |S|}$ : Matrice de transition, où chaque élément $M[i][j]$ représente la probabilité de passer de la page $i$ à la page $j$.
    \item $i, j \in \{1, 2, \ldots, |S|\}$ : Indices représentant des pages spécifiques dans le graphe, où $|S|$ est le nombre total de pages ou sommets dans le graphe.
    \item $k \in \mathbb{N}$ : Le nombre de clics (ou étapes) dans la marche.
    \item $m \in \{1, 2, \ldots, |S|\}$ : Un sommet intermédiaire dans le graphe lors de la marche.
    \item $S$ : L'ensemble des pages Web ou sommets du graphe. $S \in \mathbb{N}$ et représente le nombre total de sommets.
    \item $d^+(i) \in \mathbb{N}$ : Le demi-degré extérieur du sommet $i$, c'est-à-dire le nombre de liens sortants de $i$ dans un graphe orienté.
    \item $P(n)$ : La proposition que nous cherchons à démontrer. Spécifiquement, $P(n)$ stipule que pour tout $n \in \mathbb{N}$ et pour tout couple de sommets $(i, j) \in S \times S$, $M^n[i][j]$ donne la probabilité d'une marche de longueur $n$ de $i$ à $j$.
\end{itemize}

\subsubsection{Preuve par Induction sur $k$}

\paragraph{Proposition $P(n)$ :} Pour tout $n \geq 1$ et tout couple de sommets $(i, j) \in S \times S$, $M^n[i][j]$ est égal à la probabilité d'arriver sur la page $j$ en $n$ clics à partir de la page $i$.

$$ P(n) : \forall n \in \mathbb{N^*}, \forall i, j \in \{1, 2, \ldots, |S|\}, M^n[i][j] = \text{"probabilité d'une marche de } i \text{ à } j \text{ de longueur } n \text{"}.$$


\subsubsection{Initialisation : $P(1)$}
Par définition :

\begin{itemize}
    \item Cas 1 : $M[i][j] = \frac{1}{d^+(i)}$ si $(i, j) \in A$.
    \item Cas 2 : $M[i][j] = 0$ si $(i, j) \notin A$.
\end{itemize}

Dans les deux cas, $M^1[i][j]$ correspond aux probabilités définies pour un clic unique, donc $P(1)$ est vrai.

\subsubsection{Récurrence : $P(k) \rightarrow P(k+1)$}
Supposons que $P(k)$ soit vrai, c'est-à-dire que $M^k[i][j]$ représente la probabilité d'aller de $i$ à $j$ en $k$ clics.

Pour tout chemin de longueur $k+1$ de $i$ à $j$, il doit passer par un intermédiaire $m$. Ainsi, la probabilité d'une marche de $i$ à $j$ en $k+1$ clics est la somme sur tous les sommets intermédiaires possibles $m$ des produits des probabilités d'aller de $i$ à $m$ en $k$ clics et de $m$ à $j$ en un clic.

\[
M^{k+1}[i][j] = \sum_{m=1}^{|S|}M^k[i][m] \cdot M[m][j]
\]

Cela correspond à la définition du produit de matrices (dot-product). Par hypothèse de récurrence, $M^k[i][m]$ est la probabilité d'une marche de longueur $k$ de $i$ à $m$, et $M[m][j]$ est la probabilité d'un clic de $m$ à $j$. Ainsi, l'induction est complète et $P(n)$ est démontrée\footnote{Adapté d'une preuve consultable sur \textit{math.stackexchange.com} \cite{ramirezamaya2017proof}}.

\textbf{En Conclusion}

Pour démontrer formellement que $M^k[i][j]$ est la probabilité d'arriver sur la page $j$ en $k$ clics à partir de la page $i$, vous devrez utiliser des arguments de probabilité et la définition des marches aléatoires, en plus d'une preuve par récurrence sur la puissance de la matrice. C'est un concept fondamental en théorie des graphes et dans l'analyse des algorithmes de marche aléatoire, comme celui de PageRank.

\section{Exercice 2 : PageRank sur liste d'adjacence}

\textit{Implémentez l'algorithme PageRank sur un graphe représenté par une liste d'adjacence. Donnez le résultat de l'algorithme sur le graphe $G_1$ fourni dans le fichier \texttt{res/example\_1.txt} pour $k = 4$.}

Dans cet exercice, nous allons implémenter l'algorithme PageRank sur un graphe représenté par une liste d'adjacence. L'implémentation C de cette liste d'adjacence est fournie.

L'étape la plus cruciale de l'algorithme PageRank est bien sûr le calcul d'une rangée de score (ou vecteur de probabilités de transition à l'étape $k$). Ce calcul se base sur la cardinalité des liens sortants du sommet considéré, ainsi que sur la valeur du score de ce sommet à l'étape $k-1$.

\begin{figure}
    \centering
    \begin{tikzpicture}
        \tikzstyle{every node}=[font=\LARGE]
        \draw  (4,23.5) circle (1cm) node {\LARGE 1} ;
        \draw  (8.5,20.25) circle (1cm) node {\LARGE 2} ;
        \draw  (3.5,18.75) circle (1cm) node {\LARGE 0} ;
        \draw  (12.75,21.5) circle (1cm) node {\LARGE 4} ;
        \draw  (17.5,22.75) circle (1cm) node {\LARGE 3} ;
        \draw  (16.75,18.5) circle (1cm) node {\LARGE 5} ;
        \draw [<->, >=Stealth] (4.5,18.75) .. controls (6,19.5) and (6,19.5) .. (7.5,20);
        \draw [<->, >=Stealth] (13.75,21.75) .. controls (15.25,22.5) and (15.25,22.5) .. (16.5,23);
        \draw [<->, >=Stealth] (17,19.5) .. controls (17.25,20.75) and (17.25,20.75) .. (17.25,21.75);
        \draw [->, >=Stealth] (7.5,20.25) .. controls (6.25,21.75) and (6.25,21.75) .. (5,23.25);
        \draw [->, >=Stealth] (3.5,19.75) .. controls (3.75,21.25) and (3.75,21.25) .. (3.75,22.5);
        \draw [->, >=Stealth] (9.5,20.5) .. controls (10.75,21) and (10.75,21) .. (11.75,21.25);
        \draw [->, >=Stealth] (12.75,20.5) .. controls (14.25,19.5) and (14.25,19.5) .. (15.75,18.5);
        \draw [->, >=Stealth] (5,23.5) .. controls (8.5,22.75) and (8.5,22.75) .. (11.75,21.75);
        \node [font=\LARGE] at (11.25,24) {Graph $G_1$:};
    \end{tikzpicture}
    \caption{Représentation du graphe $G_1$ fourni dans le fichier \texttt{res/example\_1.txt}}.
    \label{fig:graph_g1}
\end{figure}

L'implémentation de cet algorithme donne ainsi le résultat suivant pour le graphe $G_1$ fourni :

\begin{minipage}{\dimexpr\linewidth-20pt}
\begin{lstlisting}[language=bash, caption={Résultat de l'algorithme PageRank sur le graphe $G_1$ fourni, pour $k = 4$.}]
    Scores at step 0: [0.166667, 0.166667, 0.166667, 0.166667, 0.166667, 0.166667]
    Scores at step 1: [0.055556, 0.138889, 0.083333, 0.250000, 0.305556, 0.166667]
    Scores at step 2: [0.027778, 0.055556, 0.027778, 0.319444, 0.291667, 0.277778]
    Scores at step 3: [0.009259, 0.023148, 0.013889, 0.423611, 0.224537, 0.305556]
    Scores at step 4: [0.004630, 0.009259, 0.004630, 0.417824, 0.239583, 0.324074]
\end{lstlisting}
\end{minipage}

Le résultat est donc, pour $k = 4$ : $[0.004630, 0.009259, 0.004630, 0.417824, 0.239583, 0.324074]$. Le sommet 3 est donc le plus important, suivi des sommets 5 et 4 à l'issue de l'étape 4.

\section{Exercice 3 : Problème de PageRank sur matrice d'adjacence}

\textit{Exécutez l'algorithme PageRank sur le graphe $G_2$ fourni dans le fichier \texttt{res/example\_2.txt} pour $k = 4$. Calculez la somme des valeurs du vecteur $s_k$ à chaque itération k. Qu'observe-t-on ? Comment évolue la somme des valeurs du vecteur $s_k$ lorsque $k$ augmente ?}

Le graphe $G_2$ fourni dans le fichier \texttt{res/example\_2.txt} est représenté dans la figure \ref{fig:graph_g2}. Il s'agit du graphe $G_1$ dans lequel on a retiré le lien $1 \rightarrow 4$.

\begin{figure}
    \centering
    \begin{tikzpicture}
        \tikzstyle{every node}=[font=\LARGE]
        \draw  (4,23.5) circle (1cm) node {\LARGE 1} ;
        \draw  (8.5,20.25) circle (1cm) node {\LARGE 2} ;
        \draw  (3.5,18.75) circle (1cm) node {\LARGE 0} ;
        \draw  (12.75,21.5) circle (1cm) node {\LARGE 4} ;
        \draw  (17.5,22.75) circle (1cm) node {\LARGE 3} ;
        \draw  (16.75,18.5) circle (1cm) node {\LARGE 5} ;
        \draw [<->, >=Stealth] (4.5,18.75) .. controls (6,19.5) and (6,19.5) .. (7.5,20);
        \draw [<->, >=Stealth] (13.75,21.75) .. controls (15.25,22.5) and (15.25,22.5) .. (16.5,23);
        \draw [<->, >=Stealth] (17,19.5) .. controls (17.25,20.75) and (17.25,20.75) .. (17.25,21.75);
        \draw [->, >=Stealth] (7.5,20.25) .. controls (6.25,21.75) and (6.25,21.75) .. (5,23.25);
        \draw [->, >=Stealth] (3.5,19.75) .. controls (3.75,21.25) and (3.75,21.25) .. (3.75,22.5);
        \draw [->, >=Stealth] (9.5,20.5) .. controls (10.75,21) and (10.75,21) .. (11.75,21.25);
        \draw [->, >=Stealth] (12.75,20.5) .. controls (14.25,19.5) and (14.25,19.5) .. (15.75,18.5);
        \node [font=\LARGE] at (11.25,24) {Graph $G_2$:};
    \end{tikzpicture}
    \caption{Représentation du graphe $G_2$ fourni dans le fichier \texttt{res/example\_2.txt}}.
    \label{fig:graph_g2}
\end{figure}

On relance l'algorithme PageRank sur le graphe $G_2$. On obtient le résultat suivant :

\begin{minipage}{\dimexpr\linewidth-20pt}
    \begin{lstlisting}[language=bash, caption={Résultat de l'algorithme PageRank sur le graphe $G_2$ fourni, pour $k = 4$, avec la somme des valeurs du vecteur $s_k$ à chaque itération $k$.}]
    S0: [0.166667, 0.166667, 0.166667, 0.166667, 0.166667, 0.166667] (sum: 1.000000)
    S1: [0.055556, 0.138889, 0.083333, 0.250000, 0.138889, 0.166667] (sum: 0.833333)
    S2: [0.027778, 0.055556, 0.027778, 0.236111, 0.152778, 0.194444] (sum: 0.694444)
    S3: [0.009259, 0.023148, 0.013889, 0.270833, 0.127315, 0.194444] (sum: 0.638889)
    S4: [0.004630, 0.009259, 0.004630, 0.258102, 0.140046, 0.199074] (sum: 0.615741)
    \end{lstlisting}
\end{minipage}

On remarque que la somme du vecteur $s_k$ diminue à chaque itération. Hors, ce vecteur est censé donné la probabilité de se trouver sur chaque sommet du graphe à l'étape $k$. La somme de ces probabilités devrait donc être égale à 1.


%%%%%%%%%%%%%%%%%%%%%%%%%%%%%%%%%%%%%%%%%%%%%%%%%%%%%%%%%%%%%%%%%%%%%%%%%%%%%%%%%%%%%%%%%

\newpage
% glossary and acronyms
% \printglossary[type=\acronymtype, title=Acronymes]
% \printglossary[title=Glossaire]
\printglossary[type=\acronymtype]

\printglossary

% biblio
\printbibliography[
    heading=bibintoc,
    category=cited,
    title={Références}
]

% uncited references (bibliography)
% https://tex.stackexchange.com/questions/6967/how-to-split-bibliography-into-works-cited-and-works-not-cited
\printbibliography[
    notcategory=cited,
    heading=bibintoc,
    title={Bibliographie additionnelle},
]

\end{document}